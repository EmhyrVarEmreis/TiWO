\documentclass[a4paper,11pt,notitlepage]{article}
\usepackage[utf8]{inputenc}	% latin2 - kodowanie iso-8859-2; cp1250 - kodowanie windows
\usepackage[T1]{fontenc}
\usepackage[polish]{babel}
\usepackage[MeX]{polski}
\selectlanguage{polish}

\usepackage{graphicx}

\hyphenation{FreeBSD}

\author{Mateusz Bocheński\\Paweł Kłapsa\\Jacek Kozieja\\Mateusz Stefaniak}
\title{Testowanie i weryfikacja oprogramowania \\ {\small Projekt 1 - Baza danych}}
\date{\today}

\linespread{1.3}

\usepackage{indentfirst}

\begin{document}
\maketitle
\tableofcontents

\section{Wstęp}
Celem projektu byłu przetestowanie operacji CRUD (Create, Read, Update, Delete) w autorskiej aplikacji Java, porozumiewającej się z bazą danych MySQL przy pomocy frameworka Hibernate. Aplikacja przetwarza dane dotyczące sklepów i znajdujących się weń produktów.
\section{Testy}
\subsection{Plan testów}
Testy miały obejmować zapis nowego obiektu w bazie danych, nadpisanie go, odczytanie i usunięcie. Przetestowano działanie metod dla:
\begin{itemize}
  \item poprawnych danych
  \item danych niezgodnych z formatem
  \item danych nie mapowanych do obiektów poprzez Hibernate
  \item danych o niepoprawnym ID (już występującym)
\end{itemize}

\subsection{Projekt i wykonanie testów}
Utworzone zostały klasy testowe: ShopCRUDTest, ProductCRUDTest oraz klasa abstrakcyjna AbstractCRUDTest<T>. Powstały metody testowe:
\begin{itemize}
  \item saveGetDeleteTest()
  \item saveListDeleteTest()
  \item saveUpdateGetDeleteTest()
\end{itemize}

\begin{footnotesize}\begin{verbatim}
---Tutaj będzie kod źródłowy i komentarze
\end{verbatim}\end{footnotesize}

\section{Wnioski}
Testy wykazały

\end{document}
