\documentclass[a4paper,11pt,notitlepage]{article}
\usepackage[utf8]{inputenc}	% latin2 - kodowanie iso-8859-2; cp1250 - kodowanie windows
\usepackage[T1]{fontenc}
\usepackage[polish]{babel}
\usepackage[MeX]{polski}
\selectlanguage{polish}

\usepackage{graphicx}

\hyphenation{FreeBSD}

\author{Mateusz Bocheński\\Paweł Kłapsa\\Jacek Kozieja\\Mateusz Stefaniak}
\title{Testowanie i weryfikacja oprogramowania \\ {\small Projekt 1 - Baza danych "Sklep"}}
\date{\today}

\linespread{1.3}

\usepackage{indentfirst}

\begin{document}
\maketitle
\tableofcontents

\section{Wstęp}
Celem projektu byłu przetestowanie operacji CRUD (Create, Read, Update, Delete) w autorskiej aplikacji Java, porozumiewającej się z bazą danych MySQL przy pomocy frameworka Hibernate. Aplikacja przetwarza dane dotyczące sklepów i znajdujących się weń produktów.
\section{Testy}
\subsection{Plan testów}
Testy miały obejmować zapis nowego obiektu w bazie danych, nadpisanie go, odczytanie i usunięcie. Należało zbadać działanie metod dla obydwu tabel - sklepów i produktów. Wykonano testy dla kilku klas równoważności a takeż ich kombinacji:
\begin{itemize}
  \item poprawne dane
  \item dane o niepoprawnym typie
  \item wartości skrajnie małe
  \item wartości skrajnie duże
\end{itemize}

\subsection{Projekt i wykonanie testów}
Utworzone zostały klasy testowe: ShopCRUDTest, ProductCRUDTest oraz klasa abstrakcyjna AbstractCRUDTest<T>. Jak można się domyślić dzięki przejrzysztości kodu, klasa ProductCRUDTest zawiera funkcje służące do testów obiektów i tabeli produktów. Testuje ona zatem klasy Product.java oraz ProductDao.java. Klasa ShopCRUDTest służy do testów sklepów. Testuje ona klasy Shop.java i ShopDao.java. Oczywiście, musi ona wykorzystywać także inne klasy - w założeniu sklep składa się z produktów. Jednak nie one stoją w tym wypadku w centrum zainteresowania.
\subsubsection{Testy produktów - klasa ProductCRUDTest}
\begin{itemize}
    \item createObject - funkcja pomocnicza tworząca obiekty produktów
    \begin{footnotesize}\begin{verbatim}
    protected Product createObject() {
        Product p = new Product();
        p.setName("name");
        p.setPrice(1.12);
        return p;
    }
\end{verbatim}\end{footnotesize}
    
    
    \item validate - funkcja pomocnicza porównująca produkt wyjściowy z zamierzonym. Używa funkcji assertEquals.
    \begin{footnotesize}\begin{verbatim}
private void validate(Product expected, Product acrual) {
        assertEquals(expected.getName(), acrual.getName());
        assertEquals(expected.getPrice(), acrual.getPrice());
    }
\end{verbatim}\end{footnotesize}
    
  \item saveGetDeleteTest - funkcja testująca. Tworzy nowy produkt i zapisuje go w bazie danych. Następnie wczytuje produkt z bazy danych i porównuje go z tym stworzonym na początku. Na koniec usuwa produkt z bazy danych i sprawdza czy próba wczytania go, zwróci NULL.
  \begin{footnotesize}\begin{verbatim}
@Test
    public void saveGetDeleteTest() throws CRUDOperationException {
        //Save
        Product product = createObject();
        Long id = productDao.save(product);

        //Get
        Product fromDb = productDao.get(id);
        validate(product, fromDb);

        //Delete
        productDao.delete(fromDb);
        assertNull(productDao.get(id));
    }
\end{verbatim}\end{footnotesize}
\item saveListDeleteTest - Funkcja testująca. Tworzy, zapisuje, odczytuje a na końcu usuwa LISTY produktów.
\begin{footnotesize}\begin{verbatim}
@Override
    public void saveListDeleteTest() throws CRUDOperationException {

        List<Product> list = new ArrayList<>(3);
        list.add(createObject());
        list.add(createObject());
        list.add(createObject());

        //Save list   
        List<Long> idList = productDao.saveAll(list);

        //Get list
        List<Product> listFromDb = productDao.list();
        assertEquals(list.size(), listFromDb.size());

        //Delete list
        for (Product p : list) {
            if (idList.contains(p.getId())) {
                productDao.delete(p);
            }
        }

        for (int i = 0; i < idList.size(); i++) {
            assertNull(productDao.get(idList.get(i)));
        }
    }
\end{verbatim}\end{footnotesize}

\item saveUpdateGetDeleteTest - Funkcja testująca. Tworzy produkt, zapisuje go w bazie danych i wczytuje. Porównuje wartość z początkową. Następnie tworzy nowy produkt o tym samym ID i nadpisuje stary produkt. Nstępnie znów wczytuje produkt z bazy i sprawdza czy jest to nowy. Usuwa produkt z bazy, sprawdza czy w tym miejscu występuje teraz NULL.
\begin{footnotesize}\begin{verbatim}
@Override
    public void saveUpdateGetDeleteTest() throws CRUDOperationException {

        //Save
        Product product = createObject();
        Long id = productDao.save(product);

        //Get
        validate(product, productDao.get(id));

        //Update
        Product newProduct = new Product();
        newProduct.setId(product.getId());
        newProduct.setName("Updated_name");
        newProduct.setPrice(2.7);
        productDao.update(newProduct);

        //Get after update
        Product fromDb = productDao.get(id);
        validate(newProduct, fromDb);

        //Delete
        productDao.delete(fromDb);
        assertNull(productDao.get(id));
    }
\end{verbatim}\end{footnotesize}
\end{itemize}


\subsubsection{Testy produktów - klasa ProductCRUDTest}
\begin{itemize}
\item createObject - funkcja pomocnicza tworząca nowy, pusty sklep.
\begin{footnotesize}\begin{verbatim}
    @Override
    protected Shop createObject() {
        Shop shop = new Shop();
        shop.setName("shopName");
        return shop;
    }
\end{verbatim}\end{footnotesize}
\item validate - funkcja pomocnicza, porównująca sklep wyjściowy z zamierzonym. Używa metody assertEquals. 
\begin{footnotesize}\begin{verbatim}
   private void validate(Shop expected, Shop actual) {
        assertEquals(expected.getName(), actual.getName());
        assertEquals(expected.getProducts().size(), actual.getProducts().size());
    }
\end{verbatim}\end{footnotesize}
\item createProducts - funkcja pomocnicza, tworzy kolekcję Set, zawierającą i produktów, które później będa mogły zapełnić sklep.
\begin{footnotesize}\begin{verbatim}
    private Set<Product> createProducts(int size) {
        Set<Product> products = new HashSet();
        for (int i = 0; i < size; i++) {
            Product p = new Product();
            p.setName("name");
            p.setPrice(1.3);
            products.add(p);
        }
        return products;
    }
\end{verbatim}\end{footnotesize}
\item saveGetDeleteTest - Funkcja testująca. Tworzy kolekcję produktów i sklep. Dodaje produkty do sklepu, a nastepnie zapisuje go w bazie danych. Wczytuje sklep z bazy danych i porównuje go z tym stworzonym na pczątku. Usuwa sklep z bazy danych i sprawdza czy na danym miejscu znajduje się NULL.
\begin{footnotesize}\begin{verbatim}
@Override
    public void saveGetDeleteTest() throws CRUDOperationException {
        int setSize = 2;

        //Save
        Set<Product> products = createProducts(setSize);
        Shop shop = createObject();
        shop.setProducts(products);
        Long id = shopDao.save(shop);

        //Get
        Shop fromDb = shopDao.get(id);
        validate(shop, fromDb);

        //Delete
        shopDao.delete(fromDb);
        assertNull(shopDao.get(id));
    }
\end{verbatim}\end{footnotesize}
\item saveListDeleteTest - Funkcja testująca. Tworzy, zapisuje, odczytuje a na końcu usuwa LISTY sklepów.
\begin{footnotesize}\begin{verbatim}
    @Override
    public void saveListDeleteTest() throws CRUDOperationException {
        int setSize = 2;

        List<Shop> list = new ArrayList<>(3);
        list.add(createObject());
        list.add(createObject());
        list.add(createObject());

        Set<Product> products = createProducts(setSize);
        for (Shop s : list) {
            s.setProducts(products);
        }

        //Save list
        List<Long> idList = shopDao.saveAll(list);

        //Get list
        List<Shop> listFromDb = shopDao.list();
        assertEquals(list.size(), listFromDb.size());

        //Delete list
        for (Shop s : listFromDb) {
            if (idList.contains(s.getId())) {
                shopDao.delete(s);
            }
        }
        Shop fromDb;
        for (Long id : idList) {
            fromDb = shopDao.get(id);
            assertNull(fromDb);
        }
    }
\end{verbatim}\end{footnotesize}

\item saveUpdateGetDeleteTest - Funkcja testująca. Tworzy sklep, kolekcję produktów i dodaje produkty do sklepu.  Zapisuje go w bazie danych i wczytuje. Porównuje wartość z początkową. Następnie zmienia dane sklepu, pozostawiając to samo ID i nadpisuje stary sklep. Następnie znów wczytuje produkt z bazy i sprawdza czy jest to nowy. Usuwa produkt z bazy, sprawdza czy w tym miejscu występuje teraz NULL.
\begin{footnotesize}\begin{verbatim}
 @Override
    public void saveUpdateGetDeleteTest() throws CRUDOperationException {
        int setSize = 2;

        //Save
        Set<Product> products = createProducts(setSize);
        Shop shop = createObject();
        shop.setProducts(products);
        Long id = shopDao.save(shop);

        //Get
        Shop fromDb = shopDao.get(id);
        validate(shop, fromDb);

        //Update
        shop.setName("NewShopName");
        for (Product p : products) {
            p.setName("New_name");
            p.setPrice(2.7);
        }
        shop.setProducts(products);
        shopDao.update(shop);

        Shop newShop = createObject();
        newShop.setName("NewShopName");
        newShop.setProducts(products);

        //Get after update
        fromDb = shopDao.get(id);
        validate(newShop, fromDb);

        //Delete
        shopDao.delete(fromDb);
        assertNull(shopDao.get(id));
    }
\end{verbatim}\end{footnotesize}
\end{itemize}
\subsubsection{Wyjątek testowy}
Na potrzeby testów został zdefiniowany nowy wyjątek który jest zgłaszany przez funkcje testowe w wypadku błędu:
\begin{footnotesize}\begin{verbatim}
package model;

public class CRUDOperationException extends Exception {

    public CRUDOperationException(String s, Throwable throwable) {
        super(s, throwable);
    }
}
\end{verbatim}\end{footnotesize}
\subsubsection{Pokrycie klas równoważności.}
\begin{itemize}
  \item poprawne dane - Nazwa produktu to string długości kilku, kilkunastu znaków. Cena produktu to double z przedziału 1-9999. Nazwa sklepu to string długości kilku, kilkunastu znaków. kolekcja produktów - j.w. Testy wypadły pozytywnie.
  \item dane o niepoprawnym typie - Nazwa produktu i sklepu to np. integer. Cena produktu to np. string. Wszystkie testy zwróciły błąd.
  \item wartości skrajnie małe - Puste nazwy produktu i sklepu, ceny równe 0. Nazwa sklepu o zerowej długości nie wywołuje błędu. Ceny produktów równe 0 wywołują błąd, tak samo jak zastosowanie typu integer zamiast double - testy wyłącznie przechodza liczby wpisane w formie "1.0" a nie "1".
  \item wartości skrajnie duże - Nazwy produktu i sklepu to stringi długości rzędu 1000 znaków. Ceny wykraczają poza zakres typu double. Skrajnie długie nazwy spowodowały błąd. Ceny przekraczające Double.MAX_VALUE nie spowodowały błędu - wartości "zawijają" się z powrotem na początek przedziału.
\end{itemize}

\section{Wnioski}
Testy wykazały poprawność działania programu przy standardowych, zamierzonych przypadkach użycia. Testy wykazały braki w konwersji typów. Być może warto o niej pomysleć? Proszę o dodanie automatycznej kowersji między typami string a double, a także dodanie obsługi typu integer. Testy dla skrajnych wartości częściowo powodują błędy. Proszę zwrócić uwagę na ich odpowiednią obsługę, jak chociażby informację zwrotną dla użytkownika.

\end{document}
